\documentclass{article}
\usepackage{mydefs}
\usepackage{todonotes}

\usepackage[backend=biber]{biblatex}
\addbibresource{fca.bib}

%%%

\title{\cclj\\ Exploratory Programming for FCA\\ Exercise Sheet}
\author{Daniel Borchmann}

%%%

\newcommand{\cclj}{\texttt{conexp-clj}\xspace}

\newcounter{exercise}
\newenvironment{exercise}{\bigskip\stepcounter{exercise}\noindent\begingroup\textbf{Exercise
    \theexercise}\hspace*{.3cm}}{\endgroup}
\newcommand{\hint}{\medskip\noindent\emph{Hint}: }

%%%

\begin{document}

\maketitle

\noindent The mathematical background of these exercises is described in~\cite{fca-book}.

\begin{exercise}
  Given $n \in \NN$, compute a contextual representation of the closure system of all
  closure systems on $\set{ 1, \ldots, n }$.  For $n \in \set{1,2,3,4}$, compute the
  number of formal concepts of these formal contexts.

  \hint
  \begin{itemize}
  \item A contextual representation of the lattice of closure systems on $M =
    \set{1,\ldots, n}$ is given by
    \begin{equation*}
      (\subsets{M}, \set{ A \to \set{x} \mid A \subseteq M, x \in M \setminus A }, \models),
    \end{equation*}
    where $N \models (A \to \set{x})$ if and only if $N$ respects $A \to \set{x}$.
  \item \texttt{(set-of-range 1 n)} yields the set \texttt{\#\{1, ..., n\}}
  \item \texttt{(subsets M)} yields a lazy sequence of all subsets of the set $M$.
  \item \texttt{(respects?\ M implication)} returns true if and only if \texttt{M} respects
    \texttt{implication}.
  \end{itemize}
\end{exercise}

\begin{exercise}
  Given $n \in \NN$, compute a contextual representation of the Tamari lattice on $n$
  symbols.  Draw the concept lattice of these formal contexts for suitable (\ie not too
  large) values of $n$.

  \hint Such a contextual representation can be obtained as
  \begin{equation*}
    (P, P, \set{ ((i, j), (p, q)) \mid (i,j),(p,q) \in P, i = q \text{ or not }
      (p \le i \le q \le j) })
  \end{equation*}
  where $P = \set{ (a, b) \mid a, b \in \set{ 1, \ldots, n }, a < b }$.
\end{exercise}

\pagebreak

\begin{exercise}
  Given $n \in \NN$, compute a contextual representation of the lattice of all
  permutations on $\set{ 1, \ldots, n }$.

  \hint If $\con K_n$ is such a contextual representation, then it can be obtained by
  \begin{align*}
    \con K_0 := \con L_0 & :=
    \begin{array}{|c|}
      \hline
      \times\\
      \hline
    \end{array}
    \\
    \con L_{n+1} &:=
    \begin{array}{c|c}
      \emptyset & \con L_n \\
      \hline
      \con L_n & \con L_n
    \end{array}
    \\
    \con K_{n+1} &:=
    \begin{array}{c|c}
      \con K_n & \con K_n \\
      \hline
      \con K_n & \con L_n
    \end{array}
  \end{align*}
  Contextual apposition and subposition are implemented in \texttt{context-apposition} and
  \texttt{context-subposition}, respectively.
\end{exercise}

\begin{exercise}
  Compute the number of linear extensions of the free distributive lattice on 3
  generators.

  \hint
  \begin{itemize}
  \item A contextual representation of the free distributive lattice on 3 generators can
    be obtained by
    \begin{equation*}
      \def\context{
        \begin{array}{|c|c|}
          \hline
          \phantom{\times} & \times \\\hline
          ~ & ~\\
          \hline
        \end{array}
      }
      \context\times\context\times\context
    \end{equation*}
    where $\times$ denotes the contextual product, which is implemented in \cclj as
    \texttt{context-product}.
  \item Let $\con K$ be the contraordinal scale of an ordered set $(S, \le)$, \ie $\con K
    = (S, S, {\not\le})$.  The number of linear extensions of $(S, \le)$ is then the value
    $\mu_{\BV(\con K)}(\emptyset, \emptyset'')$, where
    \begin{equation*}
      \mu_{\BV(\con K)}(A, B) :=
      \begin{cases}
        1 & \text{if } A = S,\\
        \sum_{(C,D) \gtrdot (A, B)} \mu_{\BV(\con K)}(C,D) & \text{otherwise}
      \end{cases}
    \end{equation*}
    where $\gtrdot$ denotes the neighborhood relation in $(\BV(\con K), \preceq)$.  Direct
    upper neighbors can be obtained by means of the function
    \texttt{direct-upper-concepts}.
  \end{itemize}
\end{exercise}

\printbibliography

\end{document}

%%% Local Variables: 
%%% mode: latex
%%% TeX-master: t
%%% ispell-local-dictionary: "american"
%%% End: 
